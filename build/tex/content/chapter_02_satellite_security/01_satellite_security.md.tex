\section{Vulnerabilities}
\label{vulnerabilities}

\section{Surveillance}
\label{surveillance}

\section{Jamming}
\label{jamming}

Jamming is the practise of willingfully blocking or distorting the signal by introducing noise (another meaningless signal). Satellite Jamming is internationally condemned and forbidden, but still happens in a lot of areas. Examples are Iran {\ldots}

Jamming is the mixing of the meaningful signal of the sender with another strong signal that is meaningless, so the receiver can not make anything of the original signal. It is like someone shouting through your conversation in the real world.

This can happen at two points in the process, First it can happen at the satellite, this is called \emph{orbital jamming}. Secondly it can happen at the receiver side, then it will be called \emph{Terrestrial(on earth) jamming}.
It must be said that it is hard, if not impossible, for the end user to know what type of jamming is occuring, it will look the same.

Jamming is mostly used for blocking television broadcasting. There are however also instances known where satellite telephony was jammed \{reference\}

\section{Orbital jamming}
\label{orbitaljamming}

Orbital jamming works by having a rogue groundstation that points a high power beam at the satellite, the sattelite, (being a passive attenuator) passes this on to all the end users, their equipment will show no signal because of the noise.

\subsection{Mitigations}
\label{mitigations}

There is nothing an end user can do to orbital jamming, it is a problem that resides with the satellite provider. As it affects all the users on the same channel they will notice quickly. Jamming is against ITU regulations, and countries will not openly acnowledge they are doing it, it is quite embarrassing, and when called out they sometimes just stop doing it.\{can we back that up?\} If not sometimes the provider can mitigate the problem by slightly repositioning the satellite.

\section{Terrestrial jamming}
\label{terrestrialjamming}

Terrestrial jamming happens at the receiving (end user) end. The jammer sets up a meaningless signal that distorts the original signal, but this time close to the end user. This works best in populated areas with a lot of satellite connections. Its effectiveness depends on the local circumstances, the power of the transmitters used and the placement of the satellite equipment.

\subsection{Mitigations}
\label{mitigations}

Again it might be hard to mitigate this, if your antenna is easily movable If you might be in an area where terrestrial jamming occurs it can be mitigated by placing the antenna so it can `see' the satellite but not much else of the sky, because it is blocked by surrounding buildings or walls. 

\section{Other Threats}
\label{otherthreats}
